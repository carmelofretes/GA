
\chapter{INTRODUCTION}

\section{MOTIVATION }

The emergence of the interest in elastic optical networks (EON) comes
from the constant increase in network traffic and the need to increase
the capacity and performance of the sections in the transmission networks.
At present, the transport technology used in optical networks is wavelength
multiplexing (WDM). This technology is capable of transporting multiple
channels in the same fiber, based on carriers of different wavelengths.
The implication of this technology is that channels that have a reduced
demand to the maximum supported by the granularity imposed, underutilized
resources; given this and because network traffic will be highly heterogeneous,
the flexibility in the provision of optical network resources is a
challenge. A major change in the architecture of EON is the replacement
of the fixed grid with a new flexible grid. The ITU-T is working on
the revision of a new G.694.1 standard \cite{huang2012}. 

The calculating of an elastic optical routing has two parts: (a) optical
routing operations (R), where calculations of the route between the
originating node and the destination are made through a network topology,
and, (b) selection of spectral resources on optical fibers (spectrum
assignment, SA). 

In WDM networks, the algorithms for routing planning and wavelength
assignment seek a physical route through the network and assign a
wavelength for the transport of that channel. The selection of that
wavelength is conditioned to be the same during the route of the physical
route, so that in this way it is not necessary to use wavelength converters
in any jump. This condition is called a continuity condition (continuity
constraint). In EON, apart from this condition, there is a new condition
that is that of contiguity in the spectrum (contiguity constraint).
This last condition means that the frequencies slots that occupy the
channel must be together in the spectrum. For that, the routing and
spectrum assignment (RSA) problem is more challenge than routing in
WDM networks and is the more important problem in EON management. 

For the resolution of the numerous problems that have multiple objectives,
a good meta-heuristic for this type of problems are the evolutionary
algorithms (EA - Evolutionary Algorithm). Traditional EAs are customized
to adapt to multi-objective problems, through the use of specialized
fitness functions and the introduction of methods to promote the diversity
of the solution. There are general approaches to the optimization
of multiple objectives. One is to combine the individual objective
functions in a single compound function or move all, except one of
them for the set of constraints. The next approach is to determine
a whole set of optimal Pareto solutions or a representative subset.
An optimal set of Pareto is a set of solutions that are not dominated
with respect to the others \cite{moga-rsa-dao}. This last approach
is more convenient for making decision over a set of trade-off best
solution instead of two first approaches. 

In this work, the main contribution is an approach based on a Multi-objective
Evolutionary Algorithms (MOEA) for the RSA problem, in which it is
determined that the proposed approach improves in terms of quality
from the Pareto front to the work presented in \cite{moga-rsa-dao}.
The MOEA optimizes: (a) the spectrum used, and (b) the total cost,
subject to the constraints of continuity, contiguity, and spectrum
conflict imposed by the EON layer. 

Our work is organized in the following way; in section 2 is explained
EON technologies concepts. In section 3, the Multi-objective Pareto
Front and Dominance concepts are explained. In section 4, the main
related works are discussed. In the next part (section 5), the RSA
problem is posed, in section 6, the contribution based on MOEA is
presented, while in section 7, the experimental environment are performed.
Finally in section 8, conclusions and future works are given. 

\section{OBJETIVE }

\subsection{GENERAL OBJECTIVES}
\begin{itemize}
\item Design and implement an exact model and a meta-heuristic, based on
Multi-Objective optimization of weighted sum and find the pareto set
of the best solutions to solve the RSA problem given a list of offline
demands point-to-point.
\end{itemize}

\subsection{SPECIFIC OBJECTIVES}
\begin{itemize}
\item Design and implement an exact model that obtains the optimal result
in networks of low complexity. 
\item Design and implement a meta-heuristic that obtains promising results
in more complex networks in an acceptable computational time. 
\item Compare the proposed meta-heuristic with an exact model published
in the literature.
\item Implement an Evolutionary Algorithm model to obtain optimal pareto
fronts for the RSA problem.
\item Analysis of the Evolutionary Algorithm proposed with a model published
in the literature.
\end{itemize}

\section{WORK ORGANIZATION}

The present work has been organized as follows: The first part or
Chapter 2 is structured as follows: definitions of an Elastic Optical
Network and the RSA problem are presented, we present the related
works and the pareto front concept. 

Chapter 3 presents the problem statement, where we present the mono
objective formulation of an exact model (ILP); a mono-objective metaheuristica
(MOGA) based on the weighted sum and a pure multi-objective metaheuristica
where we find the pareto set of the best solutions. 

In chapter 4 we present the experimental evidence and the results
obtained, conclusions and future work. 

Finally, we present the appendices and the bibliographical references.
