
\chapter{THEORETICAL FRAMEWORK}

\section{ELASTIC OPTICAL NETWORKS}

A network consists of the collection of nodes interconnected by links.
These links require transmission equipment, while the nodes require
switching equipment. The different developments and technological
research have shown that optics is one of the best for signal transmission,
since it can simultaneously amplify multiple wavelength signals in
a ravaged fiber connection. Therefore, an optical network is not necessarily
totally optical: the transmission is certainly optical, but the switching
could be optical, electrical, or hybrid \cite{Mukherjee2006}. 

The need to give the network a greater capacity to adapt to the needs
of transmission and increase the capacity and performance of the central
sections and as the demand for network traffic grows, the new paradigm
that we call elastic optical networks is born. We can define the EON
as an OTN (Optical Transport Network) where all the equipment and
the control plane can handle optical channels of variable bandwidth
and all the switching elements can support different granularities
in the spectrum of the channels that transmit information. The traditional
optical network based on WDM divides the spectrum into separate channels.
The separation between adjacent channels is between 50 GHz and 100
GHz which is specified by the ITU. The separation between channels
is very large and if each channel contains a low bandwidth used and
there is no traffic in that free gap, much of the spectrum is wasted.
In order to fully exploit a network, apart from making bandwidth more
flexible, it is necessary to have a network architecture that allows
the transmission of different signal formats for transmission. 

EONs introduce fixed granularity into the bandwidth of the channels
transported through the fiber. The ITU-T G.694.1, establishes a series
of fixed spectral grids, which divide the optical spectrum between
1530-1565 nm, from the C band, ranging from 12.5 GHz. (Giga Herz)
to 100 GHz, where most used are those of 50 GHz and 100 GHz \cite{itu-t-694-1}.
The important change in the EON architecture is the replacement of
the fixed grid (Fixed-grid) by a new flexible grid (Flexi-grid.) The
ITU-T is focused on the revision of a G.694.1 standard \cite{itu-t-694-1},
for a division of the flexible optical spectrum called flexi-grid,
for which the optical spectrum of the C band (1530-1565 nm) was defined,
which is divided into FS (Frequency Slots) of fixed sizes of 6.25,
12.5, 25 and 50 GHz \cite{itut-flexigrid} and in addition a central
frequency (CF, Central Frequency) is assigned to each elastic optical
path (EOP - Elastic Optical Path) that must coincide with the beginning
or the end of these slots existing differences in a fixed grid scheme
and a flexible grid scheme In the case of the fixed grid scheme, we
can observe the inefficient use of spectrum due to the fixed division
that has the 50 GHz spectrum between each CF's, and if we observe
the scheme of flexible grids can be noticed the free spectrum obtained
thanks to the fine granularity that it offers and that allows to assign
in a flexible way only the required bandwidth. Figure \ref{fixed_grid_flex_grid_figure}:
a) Fixed grid spectrum assignment scheme, b) Flexible grid spectrum
assignment scheme

The problem of RSA in Elastic Optical Networks is similar to the problem
of Routing and Wavelength Assignment (RSA) in networks based on WDM.
The difference between them (RSA and RWA) is the ability to flexibly
assign the frequency spectrum. The RSA is classified into two types:
Online/Dynamic and Offline/ Static traffic. In the case of the offline
RSA problem, the list of all transmission requests is already entered
as input, in order to proceed with the analysis and resolution with
this input data. For the RSA online problem, the analysis and resolution
is done as the requests arrive dynamically. In the first problem are
can be applied optimization strategies; while in second one are usually
developed heuristics. 

\begin{figure}
\includegraphics[scale=0.4]{G:/Genetico/LIBRO_TESIS/04.ListaFiguras/01\lyxdot rejillaFija_Flexible}

\caption{a) Fixed grid spectrum scheme, b) Flexible grid spectrum assigment
scheme}
\label{fixed_grid_flex_grid_figure}
\end{figure}


\section{ROUTING SPECTRUM ALLOCATION (RSA)}

The RSA problem can be attacked as routing resolution and allocation
of spectrum iterative together \cite{Christodoulopoulos2011}. In
this approach the problem RSA, the greatest difficulty arises, is
the large number of conditions that poses the problem. This introduces
greater computational complexity when calculating the optimal path
for each request, in turn optimizing the allocation of spectrum, which
ultimately translates into very large computing times.

The RSA problem in elastic optical networks is equivalent to the problem
RWA networks based on optical WDM networks. The difference between
these two technologies is the ability of the elastic networks to an
assignment of flexible spectrum to meet the data rate requested, where
a set of contiguous grooves of the spectrum is assigned to a connection,
while in WDM networks is flexible assigns a channel to the application
size. The assigned spectrum slots must always be together to satisfy
the constraint of contiguity of the spectrum. The following restrictions
are taken into account when calculating the routing and spectrum allocation.

\textbullet{} Restriction continuity of spectrum. That means the same
spectral allocation of resources on each link along a canal route. 

Restriction and elastic WDM networks. 

\textbullet{} Spectrum contiguity (or adjacency). Constraint ensures
that the subcarriers are adjacent to each other on a channel. 

Restriction on elastic networks. 

\textbullet{} Spectral Conflict. It is defined as spectrum allocation
for non-overlapping of different channels on the same fiber. 

Restriction on WMD and elastic networks. 

Basically RSA algorithms are concerned to allocate a contiguous fraction
of spectrum for each connection request subject to the above restrictions.
We see example in Figure \ref{rsa_problem_figure} given by \cite{Chatterjee2015},
as the constraints are met for a solution in elastic nets. A connection
request from node 1 to node 4 that requires 2 adjoining slots to transmit
data, we see the first figure in the 1-2-4 nodes, use the link 1 and
link 4 slots are available for the requirement in the link 1, but
in the link 4 there is only one slot available, then this does not
meet the condition of contiguity. The following figure shows the 1-2-3-4
node, use the link 1, link 2 and link 3, to establish a route, and
we see that in the three link's meet contiguity condition since the
slots are they found together in three links. 

\begin{figure}
\begin{centering}
\includegraphics{G:/Genetico/LIBRO_TESIS/04.ListaFiguras/04\lyxdot rsa}
\par\end{centering}
\caption{Restrictions continuity and contiguity}
\label{rsa_problem_figure}
\end{figure}


\section{RELATED WORK }

As the RSA is considered a NP-Complete problem \cite{IEEE:rsa-np-completo},
it has been treated with several techniques, exact and heuristic,
both for dynamic traffic and for static traffic. Among the exact techniques
are the ILP, while among the heuristics are optimizations with Colony
of Bees (BCO, Bee Colony Optimization) \cite{rsa:bco}, Genetic Algorithms
(GA, Genetic Algorithm) \cite{rsa:enfoque4,rsa:kshortestpath,daoGA-RSA},
among others \cite{aco-based}\cite{tabu-search}. 

Different ILP models for small instances and different heuristics
for more real scenarios have been used successfully to solve the RSA
problem. As an example we can mention in \cite{christodoulopoulos}
an ILP model was proposed to minimize the use of the spectrum to serve
a traffic matrix in an EON. The authors propose a method that divides
the problem into two sub-problems, the first is the routing and the
second is the spectrum assignment and solves them sequentially, using
a route-based approach. They also propose a heuristic algorithm that
serves the connections one by one sequentially. Then in \cite{Christodoulopoulos2011},
the authors extend their previous results including consideration
of modulation level. With this new consideration, a new problem was
defined routing, modulation level and spectrum assignment (RMLSA),
being outside the scope of this work. Other problems such as \textit{Fragmentation
Aware and Dynamic Traffic} are also not considered. Another ILP formulation
and the proof that the RSA problem is a NP-complete problem can be
found in \cite{IEEE:rsa-np-completo}. 

In \cite{Zhang2012rsa:restricciones2}, the differences between an
ILP for RWA and for RSA are exposed, as well as an algorithm complexity
analysis. In the same work two RSA algorithms are exposed. These have
a better performance than the ILP in larger networks. With these two
heuristic algorithms, the computational time was reduced, which is
considered an improvement compared with the ILP, with which it differentiates
in computation hours. 

The work proposed in \cite{moga-rsa-dao}, presents the multi-objective
RSA problem and its associated algorithm model. Each request has many
possible routes, and in each routing it has several spectrum assignment
options. The problem is to minimize the spectrum width to support
all requests and minimize the overall cost of the spectrum in the
link. 

The objective function for the work proposed in \cite{moga-rsa-dao}
is as following: there are two objectives associated with each solution.
The first objective $f_{1}$, is the width of the spectrum that indicates
the maximum indexed slice used in the network. The second objective
$f_{2}$ is the total cost of the spectrum link. Given a set of requests,
the route and channel are calculated for each one. After attending
each demand sequentially and without any sort of ordering, the spectrum
availabilities vector of each link is updated. 

In this work it is developed a pure multi-objective approach to calculate
a Pareto front. This approach is an extension of the work presented
in \cite{engopt} which has an approach based on weighted sum. In
our work, as in \cite{moga-rsa-dao} it has many possible routes,
and in each routing it has several spectrum assignment options. The
problem is to minimize the spectrum used and the overall cost of the
link spectrum at the same time. The same objective function is taken
from \cite{moga-rsa-dao} and the requests are handled as follows:
applications are ordered from highest to lowest, defined by the highest
possible cost of said request, the first 30\% of said list is attended
in the first place, while the remaining 70\% is treated in a random
manner, unlike \cite{moga-rsa-dao} it is a random ordering. More
details are given in section 7. 

\section{PARETO FRONT CONCEPTS }

In this section we define the concept of dominance and Pareto front
for multi-objective problem solutions. It is said that the solutions
of a problem with multiple objectives are optimal because no other
solution is superior to them when all the objectives and restrictions
are taken into account at the same time. It can be said that no objective
can be improved without degrading the other objectives. 

The set of optimal solutions is known as Pareto Optimal solutions,
in which they have multiple objectives to meet and present conflicts
when performing the simultaneous optimization of them. From this concept,
it is established as a requirement to affirm that one situation is
better than the other, which it does not diminish in anyone, but improve
at one; that is to say that one situation will be better than another,
only if in the new one it is possible to compensate the losses of
all the injured parties. In Figure \ref{pareto_figure}, you can see
the optimal Pareto sets for different scenarios with two objectives
and for the same solution space. In any case, Pareto's optimum is
always composed of solutions located at the edge of the feasible region
of the solution space.

Pareto Dominance in a context of minimization says (Min-Min Figure
\ref{pareto_figure}): that a solution $x^{1}$ dominates another
solution $x^{2}$if the following conditions are met: 1) the solution
$x^{1}$is not worse than $x^{2}$ in all the objectives. 2) The solution
$x^{1}$ is strictly better than $x^{2}$in at least one objective.
In Multi-objective Optimization is seeking to calculate the set of
non-dominate solutions on the edge of the feasible region. 

\begin{figure}
\begin{centering}
\includegraphics[scale=0.5]{G:/Genetico/LIBRO_TESIS/04.ListaFiguras/02\lyxdot ParetoFront}
\par\end{centering}
\caption{Optimal Pareto Fronts for the same solution space in four situations
of optimization with two objectives.}
\label{pareto_figure}
\end{figure}

