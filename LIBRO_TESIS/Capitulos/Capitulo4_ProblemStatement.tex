
\chapter{PROBLEM STATEMENT }

Given the physical topology, the matrix of demands and a list of pre-calculated
routes (as K-shortest path), we need to satisfy all the demands of
source-destination connection; i.e. to determine the route and spectrum
assignment for each traffic demand with optimum spectrum utilization
and he total cost. The spectrum utilization is given by the maximum
index FS used on all fibers in the network while the total cost is
depending on the distance traveled and the FS requested. 

For the proposed model, the following assumptions are established:
The spectral resource of each optical fiber is divided into FS; the
capacity of the fiber in terms of FS is limited in all links; the
connection demands are bidirectional, and a complete end-to-end optical
path must be found for each demand; A set of K specific route is given
for a connection in advance; the request is represented by three tuples
$\left(s,\,d,\,\alpha_{sd}\right)$, including the source node $s$,
the destination node $d$, and the bandwidth / data rate demanded
$\alpha$considered in the quantity of FS requested.

\section{Multi-objective Integer Linear Programming}

\begin{tabular}{lcl}
\multicolumn{3}{l}{Given:}\tabularnewline
$G$ & : & Network topology, which represents an EON\tabularnewline
$E$ & : & Set of links, in G\tabularnewline
$V$ & : & Set of vertices, in G\tabularnewline
$GB$ & : & Amount of FS for Band Guard\tabularnewline
$F_{total}$ & : & Amount of FS available in each fiber\tabularnewline
$P$ & : & Set of K routes for each demand\tabularnewline
$K$ & : & Number of available routes\tabularnewline
$SD$ & : & Quantity of demands\tabularnewline
\end{tabular}

\phantom{}

The notations and the formulation are presented below:

\phantom{}

Constants:

\begin{tabular}{lcl}
$dist\_max$ & : & Maximum distance traveled considering the longest routes. \tabularnewline
$espectrum\_max$ & : & Maximum FS index available.\tabularnewline
$cost\_max$ & : & Total cost of applications considering their maximum distances.\tabularnewline
$dist_{p}^{sd}$ & : & Distance of the route $p$\tabularnewline
$\alpha_{sd}$ & : & Quantity of FS requested by the application where $s,\,d,\,\in\,V$ \tabularnewline
\end{tabular}

\phantom{}

Indexes:

\begin{tabular}{lcl}
$sd$ & : & Demand index, $sd\,\in\,\left\{ 1,\,2,\,\ldots\,SD\right\} $\tabularnewline
$p$ & : & Route index, $p\,\in\,\left\{ 1,\,2,\,\ldots\,SD\right\} $\tabularnewline
$mn$ & : & Directional link index, $m\,\neq\,n$\tabularnewline
\end{tabular}

\phantom{}

Variables:

\begin{tabular}{lcl}
$x_{p}^{sd}$ & : & 1 if the path $p$ is used to meet the request $sd$, Otherwise \tabularnewline
$\wedge_{sd}$ & : & First FS assigned to the request $sd,\,sd\,\in\,\left\{ 0,\,\ldots,\,F_{total}-1\right\} $\tabularnewline
$\wedge_{sd,s'd'}$ & : & Indicator that is equal to 0 if $\wedge_{s'd'}$ < $\wedge_{sd}$,
and 1 in otherwise. \tabularnewline
\end{tabular}

\phantom{}

Objetive function:

\[
\;
\]

$Minimize\,f\left(x\right)\,=\,\left[f_{1}\,+\,f_{2}\,+\,f_{3}\right]$

\begin{equation}
Minimize\,f\left(x\right)\,=\,\left[f_{1}\,+\,f_{2}\,+\,f_{3}\right]\label{eq:func_obj}
\end{equation}

\[
\;
\]

Subject to:

\[
\:
\]

\begin{itemize}
\item The Spectrum use:
\end{itemize}
\begin{equation}
f_{1}\,=\,\frac{\underset{\forall sd}{max}\left(\wedge_{sd}\,+\,\alpha_{sd}\right)}{spectrum\_max}\label{eq:uso_espectro_func}
\end{equation}

\begin{itemize}
\item The total cost:
\end{itemize}
\begin{equation}
f_{2}\,=\,\frac{\sum_{sd}\sum_{p}\left(\alpha_{p}^{sd}\,*\,dist_{p}^{sd}\,*\,x_{p}^{sd}\right)}{cost\_max}\label{eq:total_costo_func}
\end{equation}

\phantom{}
\begin{itemize}
\item The distance:
\end{itemize}
\begin{equation}
f_{3}\,=\,\frac{\sum_{sd}\sum_{p}\left(dist_{p}^{sd}\,*\,x_{p}^{sd}\right)}{dist\_max}\label{eq:distancia_max_func}
\end{equation}

\phantom{}

\begin{equation}
\sum_{p\in P_{sd}}x_{p}\,=\,1,\,\forall\left(s,d\right)\label{eq:restriccion1}
\end{equation}

\phantom{}

\begin{equation}
\wedge{}_{sd}\,+\,\alpha_{sd}\,*\,x_{p}^{sd}\,+\,GB\,-\,\wedge{}_{s'd''}\,\leq\,\left(F_{total}\,+\,GB\right)\,*\,\left(1\,-\,\delta_{sd,s'd'}\,+\,2\,-\,x_{p}^{sd}\,-\,x_{p'}^{s'd''}\right)\label{eq:restriccion2}
\end{equation}

\phantom{}

\begin{equation}
\wedge{}_{s'd''}\,+\,\alpha_{s'd''}\,*\,x_{p'}^{s'd''}\,+\,GB\,-\,\wedge{}_{sd'}\,\leq\,\left(F_{total}\,+\,GB\right)\,*\,\left(1\,-\,\delta_{s'd'',sd}\,+\,2\,-\,x_{p}^{sd}\,-\,x_{p'}^{s'd''}\right)\label{eq:restriccion3}
\end{equation}

\phantom{}

\begin{equation}
\delta_{sd,s'd''}\,+\,\delta_{s'd'',sd}\,=\,1\label{eq:restriccion4_func}
\end{equation}

\phantom{}

\begin{equation}
\wedge{}_{s'd''}\,-\,\wedge{}_{sd}\,<\,F_{total}\,*\,\delta_{sd,s'd''}\label{eq:restriccion5_func}
\end{equation}

\begin{equation}
\wedge{}_{sd}\,-\,\wedge{}_{s'd''}\,<\,F_{total}\,*\text{\,}\delta_{sd,s'd''}\label{eq:restriccion6_func}
\end{equation}

\phantom{}

The objective function \ref{eq:func_obj} represents the maximum spectrum
used. The constraints \ref{eq:uso_espectro_func} represents the maximum
spectrum used, the constraints \ref{eq:total_costo_func} represents
the total cost and the constraints \ref{eq:distancia_max_func} represents
the distance traveled.

On the other hand, we have that, for all request $sd,s'd'$ and the
paths$p\,\epsilon\,P_{sd}$ and $p\,\epsilon\,P_{s'd''}$ with $p$
and $p\text{'}$sharing at least one common link $mn$ the constraints
\ref{eq:restriccion1}, \ref{eq:restriccion2}, \ref{eq:restriccion3},
\ref{eq:restriccion4_func}, \ref{eq:restriccion5_func} and \ref{eq:restriccion6_func}
ensure compliance with physical layer restrictions. 

Restrictions \ref{eq:restriccion1}, \ref{eq:restriccion2} and \ref{eq:restriccion3}ensure
that the portions of spectrum that are assigned to connections that
use paths that share a common link do not overlap and are adjacent. 

Also, for all requests $sd\,s'd'$that have $p\,\epsilon\,P_{s'd''}$,
with $p$ and $p\text{'}$ sharing at least one common link $\left(\exists\,mn\,\colon\,nm\,\epsilon\,p\,\bigwedge\,mn\,\epsilon\,p'\right)$,
the constraints \ref{eq:restriccion4_func}, \ref{eq:restriccion5_func}
and \ref{eq:restriccion6_func} ensure that either $\delta_{sd,s'd''}\,=1$
means that the initial frequency $\land_{sd}$ is smaller than the
initial frequency $\land_{s'd'}$, that is, $\land_{sd}\,<\land_{s'd''}$,
o $\delta_{s'd'',sd}\,=1$ , in which case $\land_{sd}>\land_{s'd''}$.
Note that $\land_{sd}$ and $\land_{s'd''}$ are always bounded superiorly
by $F_{total}$, and that therefore their difference will always be
less than $F_{total}$ 

\section{Multi-objective Genetic Algorimth (MOGA)}

The MOGA algorithm begins with the creation of the initial population.
The best solutions are found over several generations. Operators such
as crossing and mutation explore other possible solutions. In our
approach, not all individuals are viable solutions, therefore, additional
restrictions management procedures are required. When the stopping
criterion is met, a relatively good solution is found.

In this implementation, the objective is to find the route and the
set of FS for each request such that the total distance traveled,
the maximum FS used and the total cost are minimized and at the same
time comply with the RSA restrictions.

The parts of the implementation of the MOGA proposed in \cite{engopt}
are described in detail, given in the Algorithms, below.

The parts of the implementation of the MOGA proposed in \cite{engopt}
are described in detail, given in Algorithms \ref{moga_alg1}, \ref{moga_evaluacionPoblacion}
y \ref{moga_evaluacionIndividuo}, are described in detail below.

\begin{algorithm}[H]
\caption{MOGA}

\begin{lstlisting}
INPUT: Route table P; Total amount of FS; List of demands; 
Size of the population; Probability of mutation; 
Stop Criterion; FS Assignment Algorithm; Total Distance, 
Maximum FS, Maximum Cost 
OUTPUT: Best solution
 1: Initialize Population (P) 
 2: Evaluate Population (P) 
 3: While the stopping criterion is not met 
 4: 	P' = Select Parents (P) 
 5: 	N  = Cross (P') 
 6: 	N' = Mutar (N) 
 7: 	S  = Spectrum Assignment (N') 
 8: 	S' = Evaluate Population (S) 
 9: 	P  = Select Best Individuals (S', P) 
10: end while 
11: Return Better Solution (P)
\end{lstlisting}
\label{moga_alg1}
\end{algorithm}

\begin{algorithm}[H]
\caption{Population Evaluation}

\begin{lstlisting}
INPUT: Population P 
OUTPUT: Population evaluated 
1: for each Individual belonging P do 
2: 	Fitness = EvaluateIndividual (Individual) 
3: 	UpdateFitness (Individual, Fitness) 
4: end for 
5: Return Population
\end{lstlisting}
\label{moga_evaluacionPoblacion}
\end{algorithm}

\begin{algorithm}[H]
\caption{Evaluation of Individual}

\begin{lstlisting}
INPUT: Individual; Maximum distance; FS Maximo; 
Maximum Cost; Route table P 
OUTPUT: Fitness f; Distance f1; Spectrum f2, Costo f3 
 1: Distance = 0 
 2: FSMayor = 0 
 3: for Gen belonging Individual to do 
 4: 	Distance = Distance + Route Distance (Gen, P) 
 5: 	if FSMayor <= UltimoFS (Gen) then 
 6: 		FSMayor = UltimoFS (Gen) 
 7: 	endif
 8: 	Cost = Cost + Cost (Gen, P) 
 9: end for
10: f1 = Distance / Maximum Distance 
11: f2 = FSMayor / FS Maximo 
12: f3 = Cost / Maximum Cost 
13: f = f1 + f2 + f3 
14: return f, f1, f2, f3
\end{lstlisting}
\label{moga_evaluacionIndividuo}
\end{algorithm}


\section{NSGA II Implementation }

Our algorithm, which is an extension of the algorithm MOEA presented
in \cite{engopt}, begins with the creation of the initial population.
This MOEA is called Non-dominated Sorting Genetic Algorithm II, NSGAII.
The best solutions are found over several generations. Operators such
as crossing and mutation explore other possible solutions. 

In this implementation, the objective is to find the route and the
set of FS for each request, such that the total distance traveled,
the maximum FS used and the total cost are minimized; all this complying
with the respective RSA restrictions. The implementation of the NSGAII
is described below in Algorithm \ref{nsgaII_alg}. 
\begin{center}
\begin{algorithm}[H]
\begin{raggedright}
\caption{\textbf{NSGA II}}
\par\end{raggedright}
\begin{lstlisting}
INPUT: Route table P; Total amount of FS; List of demands; 
Size of the population; Probability of mutation; 
Stop Criterion; FS Assignment Algorithm; Total Distance, 
Maximum FS, Maximum Cost 
OUTPUT: ParetoFront 
1: Initialize Population (P) 
2: While the stop criterion is not met 
3: 	Q = generate individual (P) by selection, crossing 
		   and mutation 
4: 	Q = Q  P 
5: 	R = Construct the Pareto front from Q based in 
		   dominance 
6: 	Build Pareto fronts (R) 
7: 	Calculate Distance of Crowding (R) 
8: 	P = [0] 
9: 	while P < PopulationSize 
10: 		Include the solution in population P considering 
			Pareto ranking and Crowding Distance 
11:    End while 
12: End while 
12: return ParetoFront (P) 
\end{lstlisting}
\label{nsgaII_alg}
\end{algorithm}
\par\end{center}

In the NSGA II presented in this work, the chromosome represents a
set of requests attended. Basically, the chromosome is a compound
vector in which each gene represents an attended request. Each element
of said vector contains: the index of the assigned route (taken from
the table of pre calculated routes), and the index of the assigned
FS of the request. The steps of the algorithm procedure can be described
below:

\textbf{Initial Population.} The first step is to initialize the population.
The NSGA II begins with an initial population of chromosomes, defined
as explained below. The Algorithm deals with the requests in a determined
order, which was taken from a paper presented in \cite{capucho2013ilp}.
At work, the order is defined as follows: orders are ordered from
highest to lowest, defined by the highest possible cost of said request,
the first 30\% of said list is attended in the first place, while
the remaining 70\% is attended at random. This order is represented
by the positions of the genes in the chromosome and is maintained
throughout the execution of the algorithm. Then, randomly assign the
routes and FS to the demands, taking into account the previously defined
order. Each chromosome encodes a valid solution. 

Selection of chromosomes for the next generation. The NSGA II algorithm
shows us that the cycle begins with the selection of individuals,
in step 3. The stochastic universal sampling method is used to select
two parents to produce new individuals for the next generation \cite{rsa:enfoque3}.
Universal stochastic sampling is a sampling algorithm that is implemented
in a single phase. Given a set of n individuals and their associated
objective values, the algorithm accommodates them in a roulette wheel
where the size of the cuts assigned to each individual is proportional
to the target value. Then, a second roulette, is marked with and equally
spaced markers where and is the number of selections that you want
to make. Finally, the spinner is rotated and an individual is selected
for each marker. Position of the markers indicates the selected individual. 

\textbf{Crossover operator.} In this work we used the two-point cross
operator \cite{rsa:enfoque3} through which two cut points are randomly
generated in each player, using the same points generated, assigning
intercalary each segment generated from the parents to each child.
In algorithm 1 is applied in step 6. 

In Figure \ref{crossing_figure}, we can observe the crossing procedure
in which the cut points generated randomly were 1 and 2, dividing
the player into 3 segments. The first segment of player 1 is assigned
to the first segment of descendant 1, so the first segment of player
2 is assigned to the first segment of descendant 2. Then, the second
segment of player 1 is assigned to the second descendant, while the
second segment Player segment 2 is assigned as the second segment
of the first descendant. Then the last segments are interspersed,
resulting in both descendants shown in figure \ref{crossing_figure}.
This process is repeated until crossing the entire current population
and obtaining as a result the generation of a new population.

\textbf{Mutation.} This procedure is applied after crossing, in each
individual independently, in step 7 of algorithm 1. For the individual
selected, according to the mutation probability obtained, a position
of the vector is chosen randomly to change the route used in said
position. Selecting a route from those available for said position,
you have a higher probability of generating a feasible solution. 

\textbf{Pareto dominance.} In step 4 the union of the two populations
$Q=Q\,\cup\,P$ is performed, in step 5 and 6 the population is classified
into categories (ranking) on the basis of non-dominance. Each solution
is assigned a fitness value equal to its non-domain range (rank 0
is the best). Then the newly formed population is classified into
categories (rank) according to their domain relation, and then, as
explained in step 7, calculate the Crowding Distance of each individual,
and then select the best ones in the next cycle that begins in the
step 8, select the individuals with the best rank and crowding distance
to fill the size of the population, as seen in steps 9, 10 and 11
of algorithm 1. Therefore, the algorithm starts all over again, from
the election of breeders, until it reaches the stop condition. 

\textbf{Spectrum assignment.} A spectrum assignment algorithm is applied
to each i-th gene consecutively in the order pre-established by the
indices on the chromosome. The algorithm used in this NASGA II is
Random Fit, which randomly assigns the free FS found that complies
with the constraints of the problem. 

\textbf{Stop criterion.} A maximum execution time is used as stopping
criterion. 

\phantom{}

\begin{figure}
\includegraphics[scale=0.4]{G:/Genetico/LIBRO_TESIS/04.ListaFiguras/03\lyxdot crossOperator}

\caption{Crossing of 2 reproducers}
\label{crossing_figure}
\end{figure}

